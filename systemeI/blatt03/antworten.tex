\documentclass{scrartcl}
\usepackage{german}
\usepackage[utf8]{inputenc}
\usepackage[german]{babel}
\usepackage{array}
\usepackage{fancyhdr}
\usepackage{lastpage}
\usepackage{pgf,tikz}
\usetikzlibrary{positioning,shapes,arrows,decorations.pathreplacing,backgrounds,snakes}

\newcolumntype{x}{>{\centering\hspace{0pt}}p{1.5em}}


%%%%%%%%%%%%%%%%%%%%%%%%
% Kopf- und Fusszeilen %
%%%%%%%%%%%%%%%%%%%%%%%%
\pagestyle{fancy}
\lhead{
    \begin{tabular}{ll}
        Felix Karg \\
    \end{tabular}
}
\chead{Systeme I}
\rhead{
    \begin{tabular}{rr}
        \today{} \\
        Seite \thepage{} von \pageref{LastPage}
    \end{tabular}
}
\lfoot{}
\cfoot{}
\rfoot{}



\begin{document}

\tikzstyle{h}=[]
\tikzstyle{arrow}=[->, >=triangle 60, thick]
\tikzstyle{ve}=[rectangle split, rectangle split parts=2, draw=black, thick, text width=4cm]
\tikzstyle{inode}=[rectangle split, rectangle split parts=2, draw=black, thick, text width=4cm]
\tikzstyle{db}=[rectangle split, rectangle split parts=2, draw=black, thick, text width=5.5cm]


%%%%%%%%%%%%%%%%%%%%%%%%%%%%%%%%%%%%%%%%%%%%%%%%%%%%%%%%%%%%%%%%%%%%%%%%%%%
\section*{Antworten zu Übungsblatt Nr. 3}

\subsection*{Aufgabe 1 a)}
% linke Spalte
\begin{minipage}[t]{0.4\columnwidth}%
  \vspace{1cm}
  \textbf{FAT:}

  \begin{tabular}{c|c|}
    \cline{2-2}
    Plattenblock  0 &      \\\cline{2-2}
    Plattenblock  1 & $ 8$ \\\cline{2-2}
    Plattenblock  2 & $10$ \\\cline{2-2}
    Plattenblock  3 & $11$ \\\cline{2-2}
    Plattenblock  4 & $ 7$ \\\cline{2-2}
    Plattenblock  5 &      \\\cline{2-2}
    Plattenblock  6 & $ 3$ \\\cline{2-2}
    Plattenblock  7 & $ 2$ \\\cline{2-2}
    Plattenblock  8 & $ 9$ \\\cline{2-2}
    Plattenblock  9 & $-1$ \\\cline{2-2}
    Plattenblock 10 & $12$ \\\cline{2-2}
    Plattenblock 11 & $14$ \\\cline{2-2}
    Plattenblock 12 & $-1$ \\\cline{2-2}
    Plattenblock 13 & $ 1$ \\\cline{2-2}
    Plattenblock 14 & $-1$ \\\cline{2-2}
    Plattenblock 15 & $13$ \\\cline{2-2}
    $\vdots$        & $\vdots$\\
  \end{tabular}%
  \vspace{1cm}
\end{minipage}%
%
% rechte Spalte
\begin{minipage}[t]{0.6\columnwidth}%
  \vspace{1cm}
  \textbf{Liste freier Plattenbl\"{o}cke:}

  \begin{tabular}{|c|c|c|c|c|c|c|c}
    \hline
    5 & 0 & $\dots$\\
    \hline
  \end{tabular}

  \vspace{2cm}

  \textbf{Verzeichniseintr\"{a}ge:}

  \begin{tabular}{|c|c|c|c|c|}
    \hline
    Dateiname       & Erwei-       & Datei-    & Erster       & Datei-           \\
                    & terung       & Attribute & Plattenblock & gr\"{o}\ss{}e    \\\hline
    \texttt{BRIEF}  & \texttt{TXT} & ($\dots$) & \texttt{4}   & \texttt{129 KiB} \\
    \texttt{EDITOR} & \texttt{EXE} & ($\dots$) & \texttt{6}   & \texttt{101 KiB} \\
    \texttt{AUFGABE}& \texttt{DOC} & ($\dots$) & \texttt{15}  & \texttt{158 KiB} \\
    $\vdots$        & $\vdots$     & $\vdots$  & $\vdots$     & $\vdots$         \\
  \end{tabular}%
\end{minipage}


\subsection*{Aufgabe 1 b)}
% Dateisystemgröße Fat32 mit 28bit zur Addressierung von Blöcken und 32KiB Blockgröße, max. Dateisystemgröße?
Die maximale Dateisystemgröße ist abhängig von der Anzahl der verschiedenen Addressierbaren Blöcke
($2^{28}$) und der jeweiligen Blockgröße ($32KiB$), die größte Addressierbare größe ist also
$2^{28} * 32KiB = 8589934592KiB = 8388608MiB = 8192GiB = 8TiB.$

%%%%%%%%%%%%%%%%%%%%%%%%%%%%%%%%%%%%%%%%%%%%%%%%%%%%%%%%%%%%%%%%%%%%%%%%%%%

\subsection*{Aufgabe 2}

\begin{itemize}
  \item[a)] Hardlinks referenzieren die Datei direkt, Softlinks referenzieren einen relativen Dateipfad. \\
    Nachteile von Hardlinks: man kann keine Ordner 'verlinken', und: je nach zählweise zählt man
    so verlinkte Dateien mehrfach. \\
    Vorteile: Ein und dieselbe Datei muss nicht mehrfach existieren um vollständig referenzierbar
    bzw. verwendbar zu sein. Außerdem ist es möglich die Zugriffsrechte für alle Hardlinks
    gleichzeitig zu ändern

  \item[b)] Hardlinks zeigen direkt auf die Datei innerhalb des Dateisystems, da die Datei
  außerhalb des Dateisystems tendenziell nicht existiert ist es nicht möglich diese zu
  referenzieren.

  \item[c)] Hardlinks auf Verzeichnisse wären Problematisch, da diese im Normalfall nur eine liste
  von Hardlinks sind, und nicht im normalen speicher des Dateisystems gespeichert sind.


\end{itemize}


\subsection*{Aufgabe 3 a)}

\begin{figure}[h!]
\begin{tikzpicture}

% Header
\node[right,h] at ( 0, -0) {\textbf{Verzeichniseinträge}};
\node[right,h] at ( 5.5, -0) {\textbf{I-Nodes}};
\node[right,h] at (11, -0) {\textbf{Datenblöcke}};

% Table lines
\draw (0, -0.5) -- (16.8, -0.5);
\draw[dashed,gray] (5, 0.5) -- (5, -13);
\draw[dashed,gray] (10.5, 0.5) -- (10.5, -13);

% Verzeichniseinträge
\node[right,ve] at ( 0, -2) (VB1) {\textbf{hardlink1}\nodepart{second}\verb+brief.doc+};
\node[right,ve] at ( 0, -4) (VB4) {\textbf{hardlink2}\nodepart{second}\verb+brief.doc+};
\node[right,ve] at ( 0, -6) (VB5) {\textbf{hardlink3}\nodepart{second}\verb+brief.doc+};
\node[right,ve] at ( 0, -8) (VB2) {\textbf{brief.doc}\nodepart{second}\verb+brief.doc+};

% \node[right,ve] at ( 0, -10) (VB3) {\textbf{Hier könnte}\nodepart{second}\verb+Ihre Werbung stehen+};
\node[right,ve] at ( 0, -10) (VB6) {\textbf{symlink1}\nodepart{second}\verb+./brief.doc+};
\node[right,ve] at ( 0, -12) (VB7) {\textbf{symlink2}\nodepart{second}\verb+./brief.doc+};

% I-Nodes
\node[right,inode] at ( 5.5, -4) (IB1) {\textbf{I-Node 1}\nodepart{second}Typ: Datei, linkcount: 4};
\node[right,inode] at ( 5.5, -10) (IB2) {\textbf{I-Node 2}\nodepart{second}Typ: Link};
\node[right,inode] at ( 5.5, -12) (IB3) {\textbf{I-Node 3}\nodepart{second}Typ: Link};

% Datenblöcke
\node[right,db] at (11, -2) (DB1) {\textbf{Datenblock 1}\nodepart{second}(Anfang Inhalt von \verb+brief.doc+)};
\node[right,db] at (11, -4) (DB3) {\textbf{Datenblock 2}\nodepart{second}(weiterführung von \verb+brief.doc+)};
% \node[right,db] at (11, -10) (DB2) {\textbf{Datenblock u}\nodepart{second}\verb_/dev/zero_};

% Verbindungslinien
\draw[arrow] (VB1.east) -> (IB1.north west);
\draw[arrow] (VB4.east) -> (IB1.west);
\draw[arrow] (VB5.east) -> (IB1.south west);
%\draw[arrow] (VB1.east) -> (IB1.west);
\draw[arrow] (VB2.north east) -> (IB1.south west);
% \draw[arrow] (VB3.east) -> (IB2.west);
\draw[arrow] (IB1.east) -> (DB1.west);
% \draw[arrow] (IB2.east) -> (DB2.west);
\draw[arrow] (IB1.east) -> (DB3.west);
\draw[arrow] (VB6.east) -> (IB2.west);
\draw[arrow] (VB7.east) -> (IB3.west);
\draw[arrow] (IB2.north west) -> (VB2.south east);
\draw[arrow] (IB3.north west) -> (VB2.south east);


\end{tikzpicture}
\caption{Realisierung der Dateien im Dateisystem}
\end{figure}

\begin{itemize}

\item[b)]
Es werden mindestens 6 Datenblöcke gebraucht, also 6KiB, die allerdings vermutlich nicht vollständing belegt sind.
Insgesamt werden 4 INodes und 5 Verzeichniseinträge benötigt, das INode des Verzeichnisses mitgerechnet.


\item[c)]
Die Berechtigungen der Hardlinks und der brief.doc sind immer gleich, auch wenn bzw. nachdem sie
geändert werden. die Symlinks sind unabhängig davon, und lassen sich auch rechtemäßig nicht
einfach modifizieren.

\item[d)]
Es passiert nichts, es haben immernoch alle alle rechte, dieses verhalten ist Sinnvoll da nur indirekt auf die Datei
verlinkt ist, also die rechte der Datei selber nicht modifiziert bzw. angezeigt werden.

\end{itemize}

\subsection*{Aufgabe 4}
\begin{itemize}
\item[a)] $N_b = 10 + b / z + (b / z)^2 + (b / z)^3 $ \\
    also 10 direkt referenzierbare + ein indirekter Block der $b/z$ viele weitere direkte Adressen hat + ein
    zweifach indirekter Block der $(b/z)*(b/z)$ viele weitere Adressen beinhaltet + ein dreifach indirekter
    Block der bis zu $(b/z)^3$ Adressen bereitstellen kann.

\item[b)] Die maximale Zahl der referenzierbaren Datenblöcke beträgt
    $2^{4*8bit} = 2^{32} = 4294967296$.
    \item Blockgröße 1KiB: $10 + 256 + 256^2 + 256^3 = 16843018$ referzierbare Blöcke, also
    bis zu $16843018KiB \cong 16448MiB \cong 16GiB$ maximale Dateigröße.
    \item Blockgröße 4KiB: $(10 + 1024 + 1024^2 + 1024^3) * 4KiB = 1074791434$ Blöcke $* 4KiB \cong
    4299165736KiB \cong 4198404MiB \cong 4100GiB \cong 4TiB$, die Maximal referenzierbare
    Dateigröße beträgt also etwas über $4TiB$.

\end{itemize}

\end{document}
