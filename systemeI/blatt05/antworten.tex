\documentclass{scrartcl}
\usepackage{german}
\usepackage[utf8]{inputenc}
\usepackage[german]{babel}
\usepackage{amssymb} % math-symbol stuff
\usepackage{graphicx} % I can't do that yet
\usepackage{fancyhdr} % for header
\usepackage{lastpage} % for 'lastpage' showing
\usepackage[normalem]{ulem}  % for striking through text
\setlength{\parskip}{\medskipamount} % thats reasonable
\setlength{\parindent}{0pt} % whatever that does


%%%%%%%%%%%%%%%%%%%%%%%%
% Kopf- und Fusszeilen %
%%%%%%%%%%%%%%%%%%%%%%%%
\pagestyle{fancy}
\lhead{
    \begin{tabular}{ll}
        Felix Karg \\
    \end{tabular}
}
\chead{Systeme I}
\rhead{
    \begin{tabular}{rr}
        \today{} \\
        Seite \thepage{} von \pageref{LastPage}
    \end{tabular}
}
\lfoot{}
\cfoot{}
\rfoot{}

%%%%%%%%%%%%%%%%%%%%%%%%
% Anfang des Dokuments %
%%%%%%%%%%%%%%%%%%%%%%%%
\begin{document}

\section*{Antworten zu Übungsblatt Nr. 5}

\section*{Aufgabe 1}
\begin{itemize}
\item[a)]
    \verb$ps -e$ oder \verb$ps -A$ (sind identisch).
\item[b)]
    \verb!kill -s <n> $(pidof counter.sh)!, wobei \verb$<n>$ mit \\
    SIGSTOP = 19 \\
    SIGCONT = 18 \\
    SIGTERM = 15 \\
    SIGKILL = 9 \\
    Zu ersetzen ist.

\item[c)]
    SIGSTOP = 'Pausiere' Prozess, aber Beende ihn nicht. \\
    SIGCONT = Führe Prozess weiter, wenn 'pausiert'. \\
    SIGTERM = 'Freundlicher' Hinweis an den Prozess, dass er sich beenden soll. Kann abgefangen um individuell darauf
    zu reagieren \\
    SIGKILL = Hinweis an \sout{den Kernel} init den Prozess zu beenden, kann nicht abgefangen werden. \\

\end{itemize}


\section*{Aufgabe 2}
\begin{itemize}
\item[a)]
\begin{verbatim}
$ ps -e | wc -l
257
$ !
\end{verbatim}
\item[b)] Skript: \\
\begin{verbatim}
#! /bin/bash

echo Laufende Prozesse $(ps -e) > info.txt
echo Aktuelles Verzeichnis: $(pwd) >> info.txt
echo Erster Verzeichniseintrag: $(ls | head -n 1) >> info.txt
echo Dateiendungen: Todo ... >> info.txt

\end{verbatim}

\end{itemize}


\section*{Aufgabe 3}
\begin{itemize}
\item[a)]
Kriterium 2 wird verletzt, da eine Zeitschaltung verwendet wird. \\
Außerdem kann passieren dass Kriterium 3 verletzt wird, wenn auf der anderen Seite kein Auto wartet. \\
\item[b)]
Kriterien 3 und 4 werden Verletzt, da Alle Prozesse Blockiert werden obwohl sich keiner in seiner Kritischen Region befindet,
Und sich das in absehbarer ('endlicher', in diesem Fall sogar unendlicher) Zeit in diesem Szenario auch nicht ändern wird.
\item[c)]
Vermutlich lässt sich diskutieren ob Krit. 3 verletzt wird oder nicht, das wäre der fall wenn der 'Brückenwärter' gerade
über die Brücke 'zurückläuft' um auf der anderen Seite ein 'Auto rüberzufahren'.
Ansonsten werden alle Kriterien durchgehend erfüllt.
\item[d)]
Bei der Vorrangrichtung ist das einzige Problem, dass es Passieren könnte, dass aus der einen Richtung durchgehend
'Autos' kommen, und die andere Richtung dadurch niemals durchgelassen wird (Krit. 4 wird möglicherweise verletzt).
\item[e)]
Selbe Situation wie bei $d)$ kann entstehen, wenn von der momentan durchgelassenen Seite immer neue 'Autos' kommen.

\end{itemize}
\end{document}


Beispiel für Text, der aus einem Terminal kopiert wurde:

\begin{verbatim}
osswald@tfpool17 / $ df -h
Filesystem                           Size  Used Avail Use% Mounted on
/dev/sda4                            375G   41G  316G  12% /
dev                                  3.9G     0  3.9G   0% /dev
run                                  3.9G  480K  3.9G   1% /run
tmpfs                                3.9G     0  3.9G   0% /dev/shm
\end{verbatim}

Aufzählungen sind mit \verb_enumerate_ möglich:
\begin{enumerate}
\item Erster Punkt
\item Zweiter Punkt
\item Dritter Punkt
\end{enumerate}

\subsection*{Aufgabe 2}

Mathematische Formeln:
\begin{equation}\label{gauss}
    \sum_{i=1}^{n} i = \frac{n(n+1)}{2}
\end{equation}


Formel \ref{gauss} wird auch \emph{Gaußsche Summenformel} genannt.

Formeln können auch im Text eingebunden werden, z.B. $E = mc^{2}$.

\subsection*{Aufgabe 3}

Tabellen können mit \verb_tabular_-Umgebungen eingegeben werden:

\begin{center}
\begin{tabular}{l|l|l}
Datei             & Dateirechte        & Größe   \\
\hline
dokument.txt      & \verb_-rw-r--r--_  & 300 KB  \\
programm.exe      & \verb_-rwxr-x---_  & 450 KB  \\
mein\_verzeichnis & \verb_drwxr-xr-x_  & ---     \\
\end{tabular}
\end{center}

\end{document}


