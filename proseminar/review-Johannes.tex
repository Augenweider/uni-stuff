\documentclass{scrartcl}
\usepackage{german}
\usepackage[utf8]{inputenc}
\usepackage[german]{babel}
\usepackage{amssymb}  % advanced mathematical symbold
\usepackage{graphicx} % using graphics
\usepackage{fancyhdr} % for the head of the page
\usepackage{lastpage} % makes page numbers work
\setlength{\parskip}{\medskipamount} % thats reasonable
\setlength{\parindent}{0pt}

\usepackage{wrapfig}


%%%%%%%%%%%%%%%%%%%%%%%%
% Kopf- und Fusszeilen %
%%%%%%%%%%%%%%%%%%%%%%%%
\pagestyle{fancy}
\lhead{
    \begin{tabular}{ll}
        Felix Karg & 4342014\\
    \end{tabular}
}
\chead{Review of Using Automata to Prove ...}
\rhead{
    \begin{tabular}{rr}
        \today{} \\
        Seite \thepage{} von \pageref{LastPage}
    \end{tabular}
}
\lfoot{}
\cfoot{}
\rfoot{}

\title{Review of Using Automata to Prove Mathematical Theorems}
\author{Felix Karg}

%%%%%%%%%%%%%%%%%%%%%%%%
% Anfang des Dokuments %
%%%%%%%%%%%%%%%%%%%%%%%%
\begin{document}
\maketitle


% Review-Template
% 
% Short Overview
% Structure of the talk
% aspects of the topic you present or ignore, and why
% examples for the talk, is there sth running?
% formal and informal definitions and why
% notation
% theorems, proofs
% 
% easily understandable
% allows for someone else to follow through and roughly see the talk
% in-depth-introduction to topic

\section*{Introduction}
Sound plan. I'm looking forward to it. Maybe an aactual short introduction to
Lagrange's theorem for binary squares would have been better, but I do not know
how exactly that would be feasible.

\section*{Lagrange's Theorem for Binary Squares}
Sloppy english, as every now and then. A single read-over would certainly have
found most of the ... grammatically weird sentences, assumably being leftovers
from unfinished rewritings. I would have wanted to have additional information
on Lagrange's Theorem for Binary Squares, especially as to why it is
particularly interisting that this case is true, and not for it to appear
simply as something someone has recently found just another way of proving
this. Why is it this relevant, what are maybe some of the implications?
Maybe this is just a follow-up, but for me it's not exactly surprising that it
holds, as is. Either way, I would have wanted to have more
background-information on this Lemma, as to why it actually is relevant or
supposed to be surprising. I can understand the goal of wanting to provide a
general example of how to utilize automaton for proving number-theoretic
statements, but it's not the goal to prove something in a way just for it to be
proving in this way, as in, it's not supposed to end in itself (dt: dem
Selbstzweck dienen), right?

\section*{Using automata for proving: General approach}
In the beginning, it got explained in quite a detailed manner, still, as of now
I do not think to understand exactly what the approach is, or how I
would/should approach it. Especially the following: The Language of the
automaton is what? The functions $r: M \mapsto \Sigma^*$? How will r be
represented then? Exactly what is the goal? And why whyis it useful for us to
show that A is universal, as you put it? What does it help us to know that our
automaton will just accept every input? Would that be even valid versions of r?
Why? Additionally, but this is only something minor, why does $L(A) \subseteq
r(M)$ hold by definition?

It would be nice to know not only that you intend to contact the authors, but
that you actually did it already and are waiting wor an answer. Otherwise you
might still not have contacted them when the talk starts and that would be a
pity, since the initial process of contacting them would not take more than
30minutes at most, I'd assume.

After reading the specialization much of my confusion remains. I understand the
general procedure, here, but we aren't even creating an automaton that's going
to accept all the inputs later on, ... In the talk this can hopefully be
explained better. In step 3, we create an automaton accepting all integers with
length $\geq 18$ is that not already using the theorem, not proving it?

\section*{Construction of Automata for the Main Lemma}
I think it's a good approach for the talk.

\section*{Folded Representation of Integers}
Maybe make more explicitly clear that we need this for upholding our
constraints for the automaton as input.

\section*{Adding up}
I can nuderstand this quite well, but I think some people might be confused for
a short time regarding us constructing two automatons now, up until it is
explicitly statet. Though I would probably leave it at that. Other than that,
it might be useful to have more of a visualisation for the adding and the
ruqired transition function etc. Maybe show an example-computation, and an
example-Automaton (even a part of it) might be really helping in understanding.

\section*{Computing the Proof Using The Ultimate Framework}
Could you actually show the Code, and (if available) the verification?

\section*{General discussion}
You could get into Computer-Assisted Theorem-Provers here, explain what they do
(did they use one here?) and roughly how they work (if you have the time,
otherwise just explain how they did it).

\section*{Discussion of Didactic Questions}
I've read this part pretty much in the beginning, and other than that, my
comments are on the specific locations, if there are some.



\end{document}
