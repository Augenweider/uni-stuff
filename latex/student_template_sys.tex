\documentclass{scrartcl}
\usepackage{german}
% \usepackage[latin1]{inputenc}
\usepackage[german]{babel}
\usepackage{amssymb}
\usepackage{graphicx}
\usepackage{fancyhdr}
\usepackage{lastpage}
\setlength{\parskip}{\medskipamount}
\setlength{\parindent}{0pt}


%%%%%%%%%%%%%%%%%%%%%%%%
% Kopf- und Fusszeilen %
%%%%%%%%%%%%%%%%%%%%%%%%
\pagestyle{fancy}
\lhead{
    \begin{tabular}{ll}
        Erika Mustermann \\
        Max Müller
    \end{tabular}
}
\chead{Systeme I}
\rhead{
    \begin{tabular}{rr}
        \today{} \\
        Seite \thepage{} von \pageref{LastPage}
    \end{tabular}
}
\lfoot{}
\cfoot{}
\rfoot{} 

%%%%%%%%%%%%%%%%%%%%%%%%
% Anfang des Dokuments %
%%%%%%%%%%%%%%%%%%%%%%%%
\begin{document}

\section*{Lösungen zu Übungsblatt Nr. 0}

\subsection*{Aufgabe 1}

Beispiel fuer Text, der aus einem Terminal kopiert wurde:

\begin{verbatim}
osswald@tfpool17 / $ df -h
Filesystem                           Size  Used Avail Use% Mounted on
/dev/sda4                            375G   41G  316G  12% /
dev                                  3.9G     0  3.9G   0% /dev
run                                  3.9G  480K  3.9G   1% /run
tmpfs                                3.9G     0  3.9G   0% /dev/shm
\end{verbatim}

Aufzählungen sind mit \verb_enumerate_ möglich:
\begin{enumerate}
\item Erster Punkt
\item Zweiter Punkt
\item Dritter Punkt
\end{enumerate}

\subsection*{Aufgabe 2}

Mathematische Formeln:
\begin{equation}\label{gauss}
    \sum_{i=1}^{n} i = \frac{n(n+1)}{2}
\end{equation}


Formel \ref{gauss} wird auch \emph{Gaußsche Summenformel} genannt.

Formeln können auch im Text eingebunden werden, z.B. $E = mc^{2}$.

\subsection*{Aufgabe 3}

Tabellen können mit \verb_tabular_-Umgebungen eingegeben werden:

\begin{center}
\begin{tabular}{l|l|l}
Datei             & Dateirechte        & Größe   \\
\hline
dokument.txt      & \verb_-rw-r--r--_  & 300 KB  \\
programm.exe      & \verb_-rwxr-x---_  & 450 KB  \\
mein\_verzeichnis & \verb_drwxr-xr-x_  & ---     \\
\end{tabular}
\end{center}

\end{document}
