\documentclass{scrartcl}
\usepackage{german}
\usepackage[utf8]{inputenc}
\usepackage[german]{babel}
\usepackage{amssymb} % what does it do?
\usepackage{graphicx} % I can't do that yet
\usepackage{fancyhdr} % what does it do?
\usepackage{lastpage} % what does it do?
\setlength{\parskip}{\medskipamount} % thats reasonable
\setlength{\parindent}{0pt} % whatever that does


%%%%%%%%%%%%%%%%%%%%%%%%
% Kopf- und Fusszeilen %
%%%%%%%%%%%%%%%%%%%%%%%%
\pagestyle{fancy}
\lhead{
    \begin{tabular}{ll}
        Felix Karg \\
    \end{tabular}
}
\chead{Systeme I}
\rhead{
    \begin{tabular}{rr}
        \today{} \\
        Seite \thepage{} von \pageref{LastPage}
    \end{tabular}
}
\lfoot{}
\cfoot{}
\rfoot{}

%%%%%%%%%%%%%%%%%%%%%%%%
% Anfang des Dokuments %
%%%%%%%%%%%%%%%%%%%%%%%%
\begin{document}


\section*{Antworten zu Übungsblatt Nr. 1}

\subsection*{Aufgabe a)}
\begin{itemize}
\item cal  : Kalender
\item df   : Informationen über Dateisystem-verwendung
\item w    : Zeigt an, wer gerade auf welchem TTY seit wann angemeldet ist und was derjenige nutzer 'gerade tut'
\item id   : Zeigt informationen zu Nutzer- und Systemgruppen-Informationen zum Aktuellen Benutzer an
\item last : Ähnlich wie 'cat', zeigt aber nur den ersten Teil einer Datei an, default: 10 Zeilen
\end{itemize}

\subsection*{Aufgabe b)}
\begin{itemize}
\item uptime: Zeigt die Uhrzeit, die Aktuelle Uhrzeit, die aktuelle 'uptime',
sowie die angemeldeten Benutzer und die Load der letzten [1, 5, 15] min an.
\item date : Zeigt die Aktuelle Uhrzeit sowie das Datum an.
\item top  : Ähnlich wie htop, aber mit weniger Funktionen und farben.
Zeigt die Momentane Hardware-auslastung (Also CPU, RAM, SWAP, etc, aber ohne GPU),
und die aufteilung auf verschiedene Threads und Prozesse an.
\item uname -a : Zeigt Systeminformationen - Kernel-version, hostname, Betriebssystem,
Hardware-Plattform, Architektur und anderes - an.
\end{itemize}
Um die aktuelle Uhrzeit genau im Format \verb+"Datum: 29.10.2016, Zeit: 13:25:40"+
zu erhalten, gibt man folgenden befehl ein: \verb!date +'Datum: %d.%m.%y, Zeit: %T!
(was dasselbe ist wie \verb!date +'Datum: %d.%m.%y, Zeit: %H:%M:%S!), man verwendet also
nur einen Parameter, den Parameter +FORMAT. \\
\\
(Fals das nicht die beabsichtigte Parameter-Zählweise sein sollte, sind es entweder
\verb!%d,%m,%y,%T! oder \verb!%d,%m,%y,%H,%M,%S!.)


\subsection*{Aufgabe d)}
\begin{verbatim}
$ id
uid=46971(fk265) gid=1001(uni) Gruppen=1001(uni)
\end{verbatim}
Numerische Benutzer-id: 46971, Ich bin mitglied der Gruppe 'uni' mit der id 1001. \\
Verfügbarer RAM: "\verb!KiB Mem:   8175976 total, ..!", also 8175976KiB.

\newpage

\subsection*{Aufgabe c)}
\begin{itemize}

\item\begin{verbatim}
$ pwd         # Zeigt aktuellen absoluten pfad an (1)
/home/fk265
$ cd ..       # wechseln in das oberverzeichnis
$ pwd         # Zeigt aktuellen absoluten pfad an (2)
/home
$ ls -l       # listet alle in diesem Verzeichnis sichtbaren dateien und ordner
insgesamt 336
[...]
dr-xr-xr-x   2 root    root    0 25. Okt 22:22 es865
dr-xr-xr-x   2 root    root    0 26. Okt 10:58 fb131
dr-xr-xr-x   2 root    root    0 25. Okt 09:15 fb249
dr-xr-xr-x   2 root    root    0 25. Okt 16:18 fk255
drwx------   3 fk265   uni    82 26. Okt 17:37 fk265
dr-xr-xr-x   2 root    root    0 26. Okt 14:13 fm213
dr-xr-xr-x   2 root    root    0 26. Okt 10:21 ft48
dr-xr-xr-x   2 root    root    0 26. Okt 09:49 ge1001
[...]
$ cd          # wechseln des verzeichnisses
$ # (da kein relativer oder absoluter Pfad angegeben wird, wird nach ~ gewechselt)
$ pwd         # Zeigt aktuellen absoluten pfad an (3)
/home/fk265
$ mkdir test  # Erstelle Verzeichnis 'test' im momentanen Verzeichnis ~
$ cd /        # Wechsle nach '/'
$ pwd         # Zeigt aktuellen absoluten pfad an (4)
/
$ cd ~/test   # Wechsle zum gerade erstellten, leeren Verzeichnis
$ pwd         # Zeigt aktuellen absoluten pfad an (5)
/home/fk265/test
$
\end{verbatim}

\item\verb!ls -1  :! Zeigt alle sichtbaren Elemente eines Verzeichnisses, jedes in einer neuen Zeile. \\
\verb!ls -a  :! Zeigt wirklich Alle Elemente eines Verzeichnisses (also mit versteckten elementen, eigem Verzeichnis (.) und Oberverzeichnis (..) inklusive). \\
\verb!ls -a1 :! Zeigt alle Elemente an, jeweils in einer neuen Zeile.

\item \verb!$ find /usr/share/doc -name '*.pdf'!

\item\verb!cat  :! 'Pipe' oder 'Streame' den inhalt einer Datei an stdout. (Default: wird angezeigt) \\
\verb!more :! Einfaches Unterteilendes Anzeigen, dass man immer nur Display-Größe-Viel sieht. \\
\verb!less :! Weiterentwicklung von more, mit mehr optionen und performance-verbesserungen

\end{itemize}


\end{document}

\section*{Latex-Template}
\subsection*{Aufgabe 1}

Beispiel für Text, der aus einem Terminal kopiert wurde:

\begin{verbatim}
osswald@tfpool17 / $ df -h
Filesystem                           Size  Used Avail Use% Mounted on
/dev/sda4                            375G   41G  316G  12% /
dev                                  3.9G     0  3.9G   0% /dev
run                                  3.9G  480K  3.9G   1% /run
tmpfs                                3.9G     0  3.9G   0% /dev/shm
\end{verbatim}

Aufzählungen sind mit \verb_enumerate_ möglich:
\begin{enumerate}
\item Erster Punkt
\item Zweiter Punkt
\item Dritter Punkt
\end{enumerate}

\subsection*{Aufgabe 2}

Mathematische Formeln:
\begin{equation}\label{gauss}
    \sum_{i=1}^{n} i = \frac{n(n+1)}{2}
\end{equation}


Formel \ref{gauss} wird auch \emph{Gaußsche Summenformel} genannt.

Formeln können auch im Text eingebunden werden, z.B. $E = mc^{2}$.

\subsection*{Aufgabe 3}

Tabellen können mit \verb_tabular_-Umgebungen eingegeben werden:

\begin{center}
\begin{tabular}{l|l|l}
Datei             & Dateirechte        & Größe   \\
\hline
dokument.txt      & \verb_-rw-r--r--_  & 300 KB  \\
programm.exe      & \verb_-rwxr-x---_  & 450 KB  \\
mein\_verzeichnis & \verb_drwxr-xr-x_  & ---     \\
\end{tabular}
\end{center}

\end{document}


