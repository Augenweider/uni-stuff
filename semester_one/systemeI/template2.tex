\documentclass{scrartcl}
\usepackage{german}
\usepackage[utf8]{inputenc}
\usepackage[german]{babel}
\usepackage{array}
\usepackage{fancyhdr}
\usepackage{pgf,tikz}
\usetikzlibrary{positioning,shapes,arrows,decorations.pathreplacing,backgrounds,snakes}

\newcolumntype{x}{>{\centering\hspace{0pt}}p{1.5em}}


%%%%%%%%%%%%%%%%%%%%%%%%
% Kopf- und Fusszeilen %
%%%%%%%%%%%%%%%%%%%%%%%%
\pagestyle{fancy}
\lhead{
    \begin{tabular}{ll}
        Felix Karg \\
    \end{tabular}
}
\chead{Systeme I}
\rhead{
    \begin{tabular}{rr}
        \today{} \\
        Seite \thepage{} von \pageref{LastPage}
    \end{tabular}
}
\lfoot{}
\cfoot{}
\rfoot{}



\begin{document}

\tikzstyle{h}=[]
\tikzstyle{arrow}=[->, >=triangle 60, thick]
\tikzstyle{ve}=[rectangle split, rectangle split parts=2, draw=black, thick, text width=4cm]
\tikzstyle{inode}=[rectangle split, rectangle split parts=2, draw=black, thick, text width=4cm]
\tikzstyle{db}=[rectangle split, rectangle split parts=2, draw=black, thick, text width=5.5cm]


%%%%%%%%%%%%%%%%%%%%%%%%%%%%%%%%%%%%%%%%%%%%%%%%%%%%%%%%%%%%%%%%%%%%%%%%%%%
\section*{Aufgabe 1}
% linke Spalte
\begin{minipage}[t]{0.4\columnwidth}%
  \vspace{1cm}
  \textbf{FAT:}

  \begin{tabular}{c|c|}
    \cline{2-2}
    Plattenblock  0 &      \\\cline{2-2}
    Plattenblock  1 &      \\\cline{2-2}
    Plattenblock  2 & $10$ \\\cline{2-2}
    Plattenblock  3 & $11$ \\\cline{2-2}
    Plattenblock  4 & $ 7$ \\\cline{2-2}
    Plattenblock  5 &      \\\cline{2-2}
    Plattenblock  6 & $ 3$ \\\cline{2-2}
    Plattenblock  7 & $ 2$ \\\cline{2-2}
    Plattenblock  8 &      \\\cline{2-2}
    Plattenblock  9 &      \\\cline{2-2}
    Plattenblock 10 & $12$ \\\cline{2-2}
    Plattenblock 11 & $14$ \\\cline{2-2}
    Plattenblock 12 & $-1$ \\\cline{2-2}
    Plattenblock 13 &      \\\cline{2-2}
    Plattenblock 14 & $-1$ \\\cline{2-2}
    Plattenblock 15 &      \\\cline{2-2}
    $\vdots$        & $\vdots$\\
  \end{tabular}%
  \vspace{1cm}
\end{minipage}%
%
% rechte Spalte
\begin{minipage}[t]{0.6\columnwidth}%
  \vspace{1cm}
  \textbf{Liste freier Plattenbl\"{o}cke:}

  \begin{tabular}{|c|c|c|c|c|c|c|c}
    \hline
    15 & 13 & 1 & 8 & 9 & 5 & 0 & $\dots$\\
    \hline
  \end{tabular}

  \vspace{2cm}

  \textbf{Verzeichniseintr\"{a}ge:}

  \begin{tabular}{|c|c|c|c|c|}
    \hline
    Dateiname       & Erwei-       & Datei-    & Erster       & Datei-           \\
                    & terung       & Attribute & Plattenblock & gr\"{o}\ss{}e    \\\hline
    \texttt{BRIEF}  & \texttt{TXT} & ($\dots$) & \texttt{4}   & \texttt{129 KiB} \\
    \texttt{EDITOR} & \texttt{EXE} & ($\dots$) & \texttt{6}   & \texttt{101 KiB} \\
    $\vdots$        & $\vdots$     & $\vdots$  & $\vdots$     & $\vdots$         \\
  \end{tabular}%
\end{minipage}



%%%%%%%%%%%%%%%%%%%%%%%%%%%%%%%%%%%%%%%%%%%%%%%%%%%%%%%%%%%%%%%%%%%%%%%%%%%
\section*{Aufgabe 3}

\begin{figure}[h!]
\begin{tikzpicture}

% Header
\node[right,h] at ( 0, -0) {\textbf{Verzeichniseinträe}};
\node[right,h] at ( 5.5, -0) {\textbf{I-Nodes}};
\node[right,h] at (11, -0) {\textbf{Datenblöcke}};

% Table lines
\draw (0, -0.5) -- (16.8, -0.5);
\draw[dashed,gray] (5, 0.5) -- (5, -13);
\draw[dashed,gray] (10.5, 0.5) -- (10.5, -13);

% Verzeichniseinträge
\node[right,ve] at ( 0, -2) (VB1) {\textbf{Verzeichniseintrag 1}\nodepart{second}\verb+brief.doc+};
\node[right,ve] at ( 0, -4) (VB2) {\textbf{Verzeichniseintrag x}\nodepart{second}\verb+text+};
\node[right,ve] at ( 0, -6) (VB3) {\textbf{Hier könnte}\nodepart{second}\verb+Ihre Werbung stehen+};

% I-Nodes
\node[right,inode] at ( 5.5, -2) (IB1) {\textbf{I-Node 1}\nodepart{second}Typ: Datei};
\node[right,inode] at ( 5.5, -6) (IB2) {\textbf{I-Node z}\nodepart{second}Typ: ?};

% Datenblöcke
\node[right,db] at (11, -2) (DB1) {\textbf{Datenblock 1}\nodepart{second}(Anfang Inhalt von \verb+brief.doc+)};
\node[right,db] at (11, -6) (DB2) {\textbf{Datenblock u}\nodepart{second}\verb_/dev/zero_};

% Verbindungslinien
\draw[arrow] (VB1.east) -> (IB1.west);
\draw[arrow] (VB2.north east) -> (IB1.south west);
\draw[arrow] (VB3.east) -> (IB2.west);
\draw[arrow] (IB1.east) -> (DB1.west);
\draw[arrow] (IB2.east) -> (DB2.west);

\end{tikzpicture}
\caption{Realisierung der Dateien im Dateisystem}
\end{figure}

\pageref
\end{document}
