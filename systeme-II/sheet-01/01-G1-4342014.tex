\documentclass{scrartcl}
\usepackage{german}
\usepackage[utf8]{inputenc}
\usepackage[german]{babel}
\usepackage{amssymb} % what does it do?
\usepackage{graphicx} % I can't do that yet
\usepackage{fancyhdr} % what does it do?
\usepackage{lastpage} % what does it do?
\setlength{\parskip}{\medskipamount} % thats reasonable
\setlength{\parindent}{0pt} % whatever that does


%%%%%%%%%%%%%%%%%%%%%%%%
% Kopf- und Fusszeilen %
%%%%%%%%%%%%%%%%%%%%%%%%
\pagestyle{fancy}
\lhead{
    \begin{tabular}{ll}
        Felix Karg & 4342014\\
    \end{tabular}
}
\chead{Systeme II}
\rhead{
    \begin{tabular}{rr}
        \today{} \\
        Seite \thepage{} von \pageref{LastPage}
    \end{tabular}
}
\lfoot{}
\cfoot{}
\rfoot{}

%%%%%%%%%%%%%%%%%%%%%%%%
% Anfang des Dokuments %
%%%%%%%%%%%%%%%%%%%%%%%%
\begin{document}

\section*{Antworten zu Übungsblatt Nr. 1}

\section*{Aufgabe 1}

\begin{itemize}
    \item CSMA/CD -$>$ Transport
    \item BitTorrent (Anwendung) -$>$ Application Layer
    \item Trailer (Footer, end-of-sth-Flag) -$>$ Internet
    \item ACK -$>$ Host-to-Network
    \item Cloud -$>$ Ansichtssache. Application Layer / Internet
    \item Port 80 -$>$ Host-to-Network
    \item ipv6 -$>$ Internet
\end{itemize}

Ein paar mehr items:

\begin{itemize}
    \item Host-to-Network -$>$ Firewall
    \item Internet -$>$ DNS-Resolving
    \item Transportation -$>$ Lower-Level-Protocols (UDP, TCP, ...)
    \item Application -$>$ Protocol Buffers
    \item Host-to-Network -$>$ Usage of MAC-Adress
    \item Internet -$>$ Mesh-like Routing
    \item Transportation -$>$ Securiy (TLS)
    \item Application -$>$ Higher-Level Protocol, only seeing the necessary things
\end{itemize}



\section*{Aufgabe 2}


\begin{enumerate}
    \item (Ultra-) Schall, Licht (Laser, Glasfaser), Kapazitatorische EM-Felder, Elektrische Pulse in Datenleitungen
    \item Kein Plan wie sie diese Frage gedacht haben.
    \item Chromium (Browser), update und sync-scripts, updaten und resolven von Dependencies, verschiedene ssh-Zugriffe
    \item Ping habe ich gemacht, Spontan keine Lust auf nen Bandbreitentest. Avg Ping: 26.663 ms. Ping is Okay, Bandbreite Abhängig von eigener Bandbreite natürlich genauso (wenn mal nicht gerade irgendwelche Kritischen Switches ausfallen oder so).
\end{enumerate}

\begin{verbatim}
~> ping www.uni-freiburg.de
PING zope-vip1.ruf.uni-freiburg.de (132.230.1.159) 56(84) bytes of data.
64 bytes from zope-vip1.ruf.uni-freiburg.de (132.230.1.159): icmp_seq=1 ttl=54 time=23.8 ms
64 bytes from zope-vip1.ruf.uni-freiburg.de (132.230.1.159): icmp_seq=2 ttl=54 time=20.3 ms
64 bytes from zope-vip1.ruf.uni-freiburg.de (132.230.1.159): icmp_seq=3 ttl=54 time=39.1 ms
64 bytes from zope-vip1.ruf.uni-freiburg.de (132.230.1.159): icmp_seq=4 ttl=54 time=27.4 ms
64 bytes from zope-vip1.ruf.uni-freiburg.de (132.230.1.159): icmp_seq=5 ttl=54 time=23.5 ms
64 bytes from zope-vip1.ruf.uni-freiburg.de (132.230.1.159): icmp_seq=6 ttl=54 time=46.1 ms
64 bytes from zope-vip1.ruf.uni-freiburg.de (132.230.1.159): icmp_seq=7 ttl=54 time=21.3 ms
64 bytes from zope-vip1.ruf.uni-freiburg.de (132.230.1.159): icmp_seq=8 ttl=54 time=19.7 ms
64 bytes from zope-vip1.ruf.uni-freiburg.de (132.230.1.159): icmp_seq=9 ttl=54 time=22.9 ms
64 bytes from zope-vip1.ruf.uni-freiburg.de (132.230.1.159): icmp_seq=10 ttl=54 time=25.2 ms
64 bytes from zope-vip1.ruf.uni-freiburg.de (132.230.1.159): icmp_seq=11 ttl=54 time=24.5 ms
64 bytes from zope-vip1.ruf.uni-freiburg.de (132.230.1.159): icmp_seq=12 ttl=54 time=34.7 ms
64 bytes from zope-vip1.ruf.uni-freiburg.de (132.230.1.159): icmp_seq=13 ttl=54 time=23.1 ms
64 bytes from zope-vip1.ruf.uni-freiburg.de (132.230.1.159): icmp_seq=14 ttl=54 time=24.7 ms
64 bytes from zope-vip1.ruf.uni-freiburg.de (132.230.1.159): icmp_seq=15 ttl=54 time=22.9 ms
^C
--- zope-vip1.ruf.uni-freiburg.de ping statistics ---
15 packets transmitted, 15 received, 0% packet loss, time 14019ms
rtt min/avg/max/mdev = 19.799/26.663/46.174/7.238 ms
~>
\end{verbatim}


\end{document}


Beispiel für Text, der aus einem Terminal kopiert wurde:

\begin{verbatim}
osswald@tfpool17 / $ df -h
Filesystem                           Size  Used Avail Use% Mounted on
/dev/sda4                            375G   41G  316G  12% /
dev                                  3.9G     0  3.9G   0% /dev
run                                  3.9G  480K  3.9G   1% /run
tmpfs                                3.9G     0  3.9G   0% /dev/shm
\end{verbatim}

Aufzählungen sind mit \verb_enumerate_ möglich:
\begin{enumerate}
\item Erster Punkt
\item Zweiter Punkt
\item Dritter Punkt
\end{enumerate}

\subsection*{Aufgabe 2}

Mathematische Formeln:
\begin{equation}\label{gauss}
    \sum_{i=1}^{n} i = \frac{n(n+1)}{2}
\end{equation}


Formel \ref{gauss} wird auch \emph{Gaußsche Summenformel} genannt.

Formeln können auch im Text eingebunden werden, z.B. $E = mc^{2}$.

\subsection*{Aufgabe 3}

Tabellen können mit \verb_tabular_-Umgebungen eingegeben werden:

\begin{center}
\begin{tabular}{l|l|l}
Datei             & Dateirechte        & Größe   \\
\hline
dokument.txt      & \verb_-rw-r--r--_  & 300 KB  \\
programm.exe      & \verb_-rwxr-x---_  & 450 KB  \\
mein\_verzeichnis & \verb_drwxr-xr-x_  & ---     \\
\end{tabular}
\end{center}

\end{document}


