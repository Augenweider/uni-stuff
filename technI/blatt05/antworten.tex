\documentclass{scrartcl}
\usepackage{german}
\usepackage[T1]{fontenc}
\usepackage[latin1]{inputenc}
\usepackage[german]{babel}

% zusätzliche mathematische Symbole, AMS=American Mathematical Society
\usepackage{amssymb}
\usepackage{stmaryrd}

% fürs Einbinden von Graphiken
\usepackage{graphicx}

% für Namen etc. in Kopf- oder Fußzeile
\usepackage{fancyhdr}

% erlaubt benutzerdefinierte Kopfzeilen
\pagestyle{fancy}

% Definition der Kopfzeile
\lhead{
\begin{tabular}{ll}
Fisnik Zeqiri & 4306430 \\
Felix  Karg   & 4342014
\end{tabular}
}
\chead{}
\rhead{\today{}}
\lfoot{}
\cfoot{Seite \thepage}
\rfoot{}

\begin{document}
\section*{Antworten zum �bungsblatt Nr. 5}

\section*{Aufgabe 1}
Beh.: Die kDNF ist die Kostenminimalste DNF-Darstellung. \\ \\
Im Allgemeinen beinhaltet eine kDNF nun einen Disjunktionsterme $m*x$ sowie m�glicherweise $m*x'$.
Wenn man diese Beiden nun streicht und stattdessen nur noch $m$ innerhalb des Terms vorkommen l�sst,
beinhaltet er weniger Disjunktionsterme als die kDNF und l�sst sich dadurch zwingenderma�en g�nstiger
darstellen als diese. $ \lightning $


\section*{Aufgabe 2}

\end{document}
