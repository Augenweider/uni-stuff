\documentclass{scrartcl}
\usepackage{german}
\usepackage[utf8]{inputenc}
\usepackage[german]{babel}
\usepackage{amssymb} % what does it do?
\usepackage{graphicx} % I can't do that yet
\usepackage{fancyhdr} % what does it do?
\usepackage{lastpage} % what does it do?
\setlength{\parskip}{\medskipamount} % thats reasonable
\setlength{\parindent}{0pt} % whatever that does


%%%%%%%%%%%%%%%%%%%%%%%%
% Kopf- und Fusszeilen %
%%%%%%%%%%%%%%%%%%%%%%%%
\pagestyle{fancy}
\lhead{
    \begin{tabular}{ll}
        Felix Karg & 4342014\\
    \end{tabular}
}
\chead{Info II - AlgoDat}
\rhead{
    \begin{tabular}{rr}
        \today{} \\
        Seite \thepage{} von \pageref{LastPage}
    \end{tabular}
}
\lfoot{}
\cfoot{}
\rfoot{}

%%%%%%%%%%%%%%%%%%%%%%%%
% Anfang des Dokuments %
%%%%%%%%%%%%%%%%%%%%%%%%
\begin{document}

\section*{Antworten zu Übungsblatt Nr. 6}

\section*{Aufgabe 1}

pushBack: wenn $s_{i-1} = c_{i-1}$ dann $c_i = A * s_i$ \\
popBack: wenn $s_{i-1} \leq c_{i-1} / B$ dann $c_i = C * s_i$ \\
Bedingung: Für alle i gilt $s_i \geq f * c_i$. \\ \\
Dadurch ist offensichtlicherweise zumindest $B = 1/f$. \\
Außerdem: $A \leq B$, da sonst nach einem pushBack zu viel Freiraum vorhanden ist. \\
Gleiche Einschränkung für C aber mit Begründung über popBack, sowie für alle
natürlich dass sie größer als 1 sein sollten.


\section*{Aufgabe 3}
Wie ich soeben im Forum gelesen habe:
\begin{verbatim}
Erstmal eine Rückfrage, um Missverständnisse zu vermeiden:

Man soll laut Übungsblatt zeigen, dass n increment Operationen Laufzeit O(n) haben.

Was genau ist denn jetzt ihre Aussage? Dass Sie zeigen können, dass eine increment
Operation immer Laufzeit O(1) hat? Dann bräuchten Sie in der Tat keine amortisierte
Analyse und auch keine Potenzialfunktion. Aber dann wäre ich doch sehr neugierig,
wie Sie das zeigen bzw. wie Ihr Programm das schafft.

\end{verbatim}
(von Hannah Bast, hb1003) \\ \\

Da meine Funktion wirklich in O(1) läuft (wie nun wirklich offensichtlich sein sollte),
werde ich dementsprechend keine Analyse machen und mich darauf berufen dass
n mal eine Konstante Funktion (selbst nicht-armortisiert) insgesamt O(n) benötigt.

\end{document}

